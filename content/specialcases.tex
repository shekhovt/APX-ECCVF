\section{Complexity of Subclass Problems} \label{sec:specialcases}

% The hardness of a discrete energy minimization problem depends on the graph structure, the interaction type and the label space. 

In this section, we classify some of the special cases of energy minimization according to our complexity axis (Figure~\ref{fig:hardnessaxis}). This classification can be viewed as a reinterpretation of existing results from the literature into a unified framework.

\subsection{Class PO (Global Optimum)}
% ::: would be worthwhile to add a sentence like this:
% ** I think it's fine without it
% The first set of algorithms we consider are in class PO, and thus a global optimum may be found in polynomial time with respect to the input.

Polynomial time solvability may be achieved by considering two principal restrictions: those restricting the {\em structure} of the problem, \ie, the graph $G$, and those restricting the type of allowed interactions, \ie, functions $f_{uv}$.
%\begin{itemize}
%    \item Structure restrictions: tree, bounded tree width, outer planar.
%    \item Interaction restrictions: characterization, bsiubmodular, submodular, convex, mincut.
%    \item Unary constraints and mixed pairwise-unary constraints.
%\end{itemize}

\textbf{Structure Restrictions.} When $G$ is a chain, energy minimization
% ~\eqref{eq:1}
reduces to finding a shortest path in the trellis graph, which can be solved using a classical dynamic programming method known, \eg, as the Viterbi algorithm~\cite{forney1973viterbi}.
%the Viterbi algorithm~\cite{forney1973viterbi}.
%::: get a better cite for viterbi.  do not cite wikipedia (ever).  you are not in 4th grade.
%:::old text: a classical dynamic programming method known, \eg, as Viterbi algorithm~\cite{wikipedia}.
%## What one can learn from wikipedia is that Viterbi was only onle of the gentelmen who have proposed such algorithms in different context
%When $G$ is a tree, the same DP principle applies: given a fixed state of a node the problem decouples into independent minimizations.
The same dynamic programming (DP) principle applies to graphs of bounded tree-width.  Fixing all variables in a separator set decouples the problem into independent optimization problems. For treewidth 1, the separators are just individual vertices and the problem is solved by a variant of DP~\cite{Pearl-88,SchlesingerHlavac2002}.
For higher treewidth, the respective optimization procedure is known as junction tree decomposition~\cite{Lauritzen96}. A loop is a simple example of a treewidth 2 problem. However, for a treewidth $k$, the time complexity is exponential in $k$~\cite{Lauritzen96}.
Finally, when $G$ is an outer-planar graph, the problem can be solved by the method of~\cite{Schraudolph-10}, which reduces it to a planar Ising model, for which efficient algorithms exist~\cite{Shih-90}.


\textbf{Interaction Restrictions.}
%Binary image segmentation with attractive interactions reduces to a minimum cut problem~\cite{Greig89}.
% Binary energy minimization with attractive interactions reduces to minimum cut problem~\cite{Greig89}. The set of problems solvable in this way is closely related to submodularity.
Submodularity is a restriction closely related to problems solvable by minimum cut. A quadratic pseudo-Boolean function $f$ is {\em submodular} iff its quadratic terms are non-positive.  It is then known to be equivalent with finding a minimum cut in a corresponding network~\cite{Hammer:OR68}. 
Another way to state this condition for QPBO is 
% \begin{align} \label{eq:submodular}
$\forall (u, v) \in \E, f_{uv}(0, 1) + f_{uv}(1, 0) \geq f_{uv}(0, 0) + f_{uv}(1, 1).$
% \end{align} 
%
However, submodularity is more general.  It extends to higher-order and multi-label problems. Submodularity is considered a discrete analog of convexity. Just as convex functions are relatively easy to optimize, general submodular function minimization  can be solved in strongly polynomial time~\cite{Schrijver00}.
Kolmogorov and Zabin introduced submodularity in computer vision and showed that binary 2\textsuperscript{nd} order and 3\textsuperscript{rd} order submodular problems can be always reduced to minimum cut, which is much more efficient than general submodular function minimization~\cite{kolmogorov2004energy}. \citeauthor{Zivny-08-binary} and \citeauthor{ramalingam2008exact} give more results on functions reducible to minimum cut~\cite{Zivny-08-binary,ramalingam2008exact}.
For QPBO on an unrestricted graph structure, the following {\em dichotomy} result has been proven by~\citet{Cohen-04}:
 %\begin{theorem}[\citet{Cohen-04}]
 %Let L be a class of functions from $\{0, 1\}$ to $\Rationals$. Let $C_L$ be the
 %class of instances with functions from $L$.
 %\begin{enumerate}
 %\item Either every $f \in L$ is submodular and thus $C_L \in $ PO,
 %\item or $C_L$ encodes MAXCUT and hence is NP-hard.
 %\end{enumerate}
 %\end{theorem}
%\noindent This theorem means that 
either the problem is submodular and thus in PO or it is NP-hard (\ie, submodular problems are the only ones that are tractable in this case).
%, implying that the only tractable problems are those that are submodular.
%if and only if it is submodular\footnote{Assuming P $\neq$ NP}.
%
%\citet{kolmogorov2004energy} have studied the tractability for QPBO. They found in general QPBO is NP-hard and the necessary and sufficient condition for QPBO to be able to be solved by graph cut is the submodularity condition:
%% There might be different submodular definitions in the literature. Let's stick to the one that is widely used in the vision community.
%\begin{align} \label{eq:submodular}
%\forall (u, v) \in \E, f_{uv}(0, 1) + f_{uv}(1, 0) \geq f_{uv}(0, 0) + f_{uv}(1, 1)
%\end{align}
%
%Note that submodularity condition is {\em not the only} sufficient condition for QPBO to be in PO, as we shown later in the mixed restriction subsection. 

For multi-label problems~\citeauthor{Ishikawa03} proposed a reduction to minimum cut for problems with convex interactions, \ie, where $f_{uv}(x_u,x_v) = g_{uv}(x_u - x_v)$ and $g_{uv}$ is convex and symmetric~\cite{Ishikawa03}. 
It is worth noting that when the unary terms are convex as well, the problem can be solved even more efficiently~\cite{Hochbaum-2001-MRF,Kolmogorov05primal-dualalgorithm}. %The construction by~\citet{Hochbaum-2001-MRF} 
The same reduction~\cite{Ishikawa03} remains correct for a more general class of submodular multi-label problems.
% ** Can we introduce the order dependent nature before introducing component-wise minimum?
In modern terminology, component-wise minimum $x \wedge y$ and component-wise maximum $x \vee y$ of complete labelings $x$, $y$ for all nodes are introduced. These operations depend on the {\em order of labels} and, in turn, define a lattice on the set of labelings. The function $f$ is called {\em submodular on the lattice} if $f(x \vee y) + f(x \wedge y) \leq f(x) + f(y)$ for all $x$, $y$~\cite{Topkis-78}.
In the pairwise case, the condition can be simplified to the form of submodularity common in computer vision \cite{ramalingam2008exact}:
\begin{align} \label{eq:submodular}
f_{uv}(i, j+1) + f_{uv}(i+1, j) \geq f_{uv}(i, j) + f_{uv}(i+1, j+1).
\end{align}
In particular, it is easy to see that a convex $f_{uv}$ satisfies it~\cite{Ishikawa03}.
%equivalent to that each $f_{uv}$ is submodular as defined in~\eqref{eq:submodular}.
\citet{Kolmogorov-10} and \citet{Arora-12} proposed maxflow-like algorithms for higher order submodular energy minimization. 
\citeauthor{DSchlesinger-07-permuted} proposed an algorithm to find a reordering in which the problem is submodular if one exists~\cite{DSchlesinger-07-permuted}. 
 %Generalizing on operation $\vee$ and $\wedge$, a class of bisubmodular 
%\revisit
However, unlike in the binary case, solvable multi-label problems are more diverse. A variety of problems are generalizations of submodularity and are in PO, including symmetric tournament pair, submodularity on arbitrary trees, submodularity on arbitrary lattices, skew bisubmodularity, and bisubmodularity on arbitrary domains (see references in~\cite{Thapper-13}).
%::: previous text (not a sentence): To name a few, (see references in~\cite{Thapper-13}): symmetric tournament pair, submodularity on arbitrary trees, submodularity on arbitrary lattices, skew bisubmodularity, and bisubmodularity on arbitrary domains are all generalizations of submodularity and are all in PO.
\citet{Thapper-12} and \citet{Kolmogorov-12-LP-power} characterized these tractable classes and proved a similar dichotomy result: a of problem of unrestricted structure is either solvable by LP-relaxation (and thus in PO) or it is NP-hard. It appears that LP relaxation is the most powerful and general solving technique~\cite{Zivny-Werner-Prusa-ASP-MIT2014}.
%
%There have been many generalizations of multi-label submodular functions proposed until the following result has been finally established:
% \begin{theorem}[\citet{Thapper-12,Thapper-13}]
% Let $D$ be an arbitrary finite set and $L$ a class of functions from $D$ to $\Rationals$. Let $C_L$ be the class of instances with functions from $L$.
% \begin{enumerate}
% \item Either every $f \in L$ satisfies a certain algebraic condition
% and Basic LP relaxation (BLP) solves every instance from $C_L$,
% \item or $C_L$ encodes MAXCUT and hence is NP-hard.
% \end{enumerate}
% \end{theorem}
% It follows that all tractable problems with interaction restrictions has been characterized. Moreover, if the problem is tractable it can be solved by an LP relaxation technique.

%In case unary function are convex as well, an efficient~\citet{Hochbaum-2001-MRF} can solve the problem in time not dependent on the number of labels.
%
%In the case of more than 2 labels, the {\em submodularity} is defined given a totally ordered label space $\mathcal{L}$ \footnote{There exist alternative definitions for multi-label submodularity \cite{Kolmogorov-Rother-07-QBPO-pami}, under this definition the NP-hard metric labeling problem is considered as a submodular subclass. \cite{kappes2015comparative}}\cite{ramalingam2008exact}: $\forall (u, v) \in \E \text{ and } \forall a, b \in \mathcal{L}$,
%\begin{align} \label{eq:multisubmodular}
%f_{uv}(a+1, b) + f_{uv}(a, b+1) \geq f_{uv}(a, b) + f_{uv}(a+1, b+1)
%\end{align}
%
% TODO: the citation below might be incomplete
%Polynomial time exact inference \cite{ramalingam2008exact} is possible under this multi-label submodular conditions. In addition, \citet{DSchlesinger-07-permuted} proposed an algorithm to find a reordering of the label space $\mathcal{L}$ in which the problem is submodular if one exists. The tractability of multi-label submodular problems is studied of a more general context as valued CSP \cite{Thapper-12,Thapper-13}. The studies characterize {\em all tractable problems with interaction restrictions has been characterized and shows that if the problem is tractable it can be solved by an LP relaxation technique.}

% higher-order submodular case not discussed

\textbf{Mixed Restrictions.}
In comparison, results with mixed structure and interaction restrictions are rare. One example is a planar Ising model without unary terms~\cite{Shih-90}. Since there is a restriction on structure (planarity) and unary terms, it does not fall into any of the classes described above. 
% ** consider removing the above sentence if running out of space
Another example is the restriction to supermodular functions on a bipartite graph, solvable by~\cite{DSchlesinger-07-permuted} and by LP relaxation, but not falling under the characterization~\cite{Thapper-13} because of the graph restriction. %\citet{boykov2001approximate} uses 

\textbf{Algorithmic Applications.}
The aforementioned tractable formulations in PO can be used to solve or approximate harder problems. Such trees, cycles and planar problems are used in dual decomposition methods~\cite{Komodakis-subgradient,KomodakisP08,BatraGPC10}. Binary submodular problems are used for finding an optimized crossover of two-candidate multi-label solutions. An example of this technique, the expansion move algorithm, achieves a constant approximation ratio for the Potts model~\cite{boykov2001approximate}. Extended dynamic programming can be used to solve restricted segmentation problems~\cite{Felzenszwalb-10-Tiered} and as move-making subroutine~\cite{VineetWT12}.
LP relaxation also provides approximation guarantees for many problems~\cite{Kleinberg99,Chekuri-01,Komodakis-07,BachHG15}, placing them in the APX class.
%, as explained in the next subsection. 
%
% ** Connection to LP relaxation not explained in APX section
% So is QPBO with submodular interactions. 
% and starting from pseudo-Boolean case. Binary image segmentation with attractive constrains~\cite{Greig89} reduces to a minimum cut problem. In fact, QPBO can be always written as an extended to by negative weights minimum cut by selecting appropriate weights, but only in the case the weights are non-negative the problem is solvable (and has maximum flow dual). It turns out  
%As mentioned earlier in case of $0$-$1$ variables the problem of optimizing QPBO is known to be solvable 

% \revisit
%
%\par
%Special cases in PO are the ones that are most studied. Their popularity is a result of not only their simplicity, but also their role as the foundation to solve harder problems. First, if the topological structure forms a tree, then global optimum can always be found in polynomial time through message passing with dynamic programming. Assuming arbitrary structure but limiting the the label space to be Boolean and the problem to be submodular (defined below), we can find the global optimum through graph-cut. The algorithm has low order polynomial complexity in theory and is shown to be very efficient in practice for computer vision problems. Despite that many multi-class (more than 2 labels) problems do not belong to PO, they can still be approximated by a series of binary ones. 
%
%\begin{definition}[\cite{rother2007optimizing}]
%A binary (two-class) energy minimization problem is {\em submodular} if and only if $\forall{u,v\in\V}$
%\begin{align} \label{eq:submodular}
%f_{uv}(0, 1) + f_{uv}(1, 0) \geq f_{uv}(0, 0) + f_{uv}(1, 1)
%\end{align}
%\end{definition}
%
%Multi-class problems can in fact be in PO if we consider a totally ordered set as the label space, for instance, the set of consecutive integers. Making use of finite difference, Ishikawa \cite{ishikawa2003exact} has given an exact reduction to graph-cut from problems with convex priors with respect to a totally ordered label set:
%
%\begin{definition}[\cite{ishikawa2003exact}]
%If for each edge $uv \in \E$, pairwise potential $f_{uv}(x_u,x_v)$ takes the form $c_uvg(x_u-x_v)$, where $c_uv$ is an edge dependent constant and $g(x)$ is a {\em convex function}, then problem \ref{eq:1} is called one with {\em convex priors}.
%\end{definition}
%
%Note that the well-known Potts model is not a special case of this definition and the convexity here is defined on a discrete set. 

\subsection{Class APX (Bounded Approximation)} \label{sec:apx}

Problems that have bounded approximation in polynomial time usually have certain restriction on the interaction type. The Potts model may be the simplest and most common way to enforce the smoothness of the labeling. Each pairwise interaction depends on whether the neighboring labellings are the same, \ie $f_{uv}(x_u,x_v) = c_{uv}\delta(x_u, x_v)$.  Boykov \etal showed a reduction to this problem from the NP-hard multiway cut~\cite{boykov2001approximate}, also known to be APX-complete~\cite{ausiello1999complexity, dahlhaus1994complexity}. They also proved that their constructed alpha-expansion algorithm is guaranteed to produce a factor two approximation to the global optimum. These results prove that the Potts model is in APX but not in PO. However, their reduction from multiway cut is not an AP-reduction, as it violates the third condition of AP-reducibility. Therefore, it is still an open problem whether the Potts model is APX-complete.
% removed footnote:
%  \footnote{There is a workaround to satisfy the third condition by defining a proper mapping $\sigma$, but it is then not obvious how to satisfy the fifth condition.}.
Boykov \etal also showed that their algorithm can approximate the more general problem of metric labeling~\cite{boykov2001approximate}. The energy is called {\em metric} if, for an arbitrary, finite label space $\mathcal{L}$, the pairwise interaction satisfies a) $ f_{uv}(\alpha, \beta) = 0$, b) $f_{uv}(\alpha, \beta) = f_{uv}(\beta, \alpha) \geq 0$, and c) $f_{uv}(\alpha, \beta) \leq f_{uv}(\beta, \gamma) + f_{uv}(\beta, \gamma)$,
% \begin{align}
%         f_{uv}(\alpha, \beta) & = 0 \leftrightarrow \alpha = \beta, \\
%         f_{uv}(\alpha, \beta) & = f_{uv}(\beta, \alpha) \geq 0, \\
%         f_{uv}(\alpha, \beta) & \leq f_{uv}(\beta, \gamma) + f_{uv}(\beta, \gamma),
% \end{align}
for any labels $\alpha$, $\beta$, $\gamma \in \mathcal{L}$ and any $uv \in \E$. Although their approximation algorithm has a bound on the performance ratio, the bound depends on the ratio of some pairwise terms, a number that can grow exponentially large. For metric labeling with $k$ labels, Kleinberg \etal proposed an $O(\log k \log \log k)$-approximation algorithm. This bound was further improved to $O(\log k)$ by \citet{Chekuri:2005:LPF}, making metric labeling a problem in log-APX \footnote{Observe that $k = O(\text{poly}(|x|))$, where $|x|$ is the input instance size (see \cref{C:label-approx}). }.
 %As the label size $k$ can at most be a polynomial of the input size $|x|$ (See the proof for \cref{C:label-approx} in \cref{sec:formalproof}), the bound on the performance ratio is equivalent to $O(\log |x|)$, making metric labeling a problem in log-APX.

We have seen that a problem with convex pairwise interactions is in PO. An interesting variant is its truncated counterpart, \ie, $f_{uv}(x_u,x_v) = w_{uv}\min\{d(x_u - x_v), M\}$, where $w_{uv}$ is a non-negative weight, $d$ is a convex symmetric function to define the distance between two labels and $M$ is the truncating constant~\cite{veksler2007graph}.
This problem is NP-hard~\cite{veksler2007graph}, but \citet{Kumar-11-improved} have proposed an algorithm that yields bounded approximations with a factor of $2+\sqrt{2}$ for linear distance functions and a factor of $O(\sqrt{M})$ for quadratic distance functions\footnote{In these truncated convex problems, the ratio bound is defined for the pairwise part of the energy \cref{eq:1}. The approximation ratio in accordance to our definition is obtained assuming the unary terms are non-negative.}. This bound is analyzed for more general distance functions by~\citet{kumar2014rounding}.

Another APX problem with implicit restrictions on the interaction type is logic MRF~\cite{bach2015unifying}. It is a powerful higher order model able to encode arbitrary logical relations of Boolean variables. It has energy function $f(x) = \sum_i^n w_iC_i$, where each $C_i$ is a disjunctive clause involving a subset of Boolean variables $x$, and $C_i = 1$ if it is satisfied and 0 otherwise. Each clause $C_i$ is assigned a non-negative weight $w_i$. The goal is to find an assignment of $x$ to maximize $f(x)$. As disjunctive clauses can be converted into polynomials, this is essentially a pseudo-Boolean optimization problem. However, this is a special case of general 2-label energy minimization, as its polynomial basis spans a subspace of the basis of the latter. \citet{bach2015unifying} proved that logic MRF is in APX by showing that it is a special case of MAX-SAT with non-negative weights.
%
%\citeauthor{bach2015unifying} showed that logic MRF is a special case of MAX-SAT, which admits a constant factor approximation algorithm in polynomial time~\cite{bach2015unifying}. This implies that the same algorithm can be used to solve logic MRF with a approximation guarantee and, thus, logic MRF is in APX. 

